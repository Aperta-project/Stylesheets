\documentclass[11pt,twoside]{article}\makeatletter

\IfFileExists{xcolor.sty}%
  {\RequirePackage{xcolor}}%
  {\RequirePackage{color}}
\usepackage{colortbl}
\usepackage{wrapfig}
\usepackage{ifxetex}
\ifxetex
  \usepackage{fontspec}
  \usepackage{xunicode}
  \catcode`⃥=\active \def⃥{\textbackslash}
  \catcode`❴=\active \def❴{\{}
  \catcode`❵=\active \def❵{\}}
  \def\textJapanese{\fontspec{Kochi Mincho}}
  \def\textChinese{\fontspec{HAN NOM A}\XeTeXlinebreaklocale "zh"\XeTeXlinebreakskip = 0pt plus 1pt }
  \def\textKorean{\fontspec{Baekmuk Gulim} }
  \setmonofont{DejaVu Sans Mono}
  
\else
  \IfFileExists{utf8x.def}%
   {\usepackage[utf8x]{inputenc}
      \PrerenderUnicode{–}
    }%
   {\usepackage[utf8]{inputenc}}
  \usepackage[english]{babel}
  \usepackage[T1]{fontenc}
  \usepackage{float}
  \usepackage[]{ucs}
  \uc@dclc{8421}{default}{\textbackslash }
  \uc@dclc{10100}{default}{\{}
  \uc@dclc{10101}{default}{\}}
  \uc@dclc{8491}{default}{\AA{}}
  \uc@dclc{8239}{default}{\,}
  \uc@dclc{20154}{default}{ }
  \uc@dclc{10148}{default}{>}
  \def\textschwa{\rotatebox{-90}{e}}
  \def\textJapanese{}
  \def\textChinese{}
  \IfFileExists{tipa.sty}{\usepackage{tipa}}{}
  \usepackage{times}
\fi
\def\exampleFont{\ttfamily\small}
\DeclareTextSymbol{\textpi}{OML}{25}
\usepackage{relsize}
\RequirePackage{array}
\def\@testpach{\@chclass
 \ifnum \@lastchclass=6 \@ne \@chnum \@ne \else
  \ifnum \@lastchclass=7 5 \else
   \ifnum \@lastchclass=8 \tw@ \else
    \ifnum \@lastchclass=9 \thr@@
   \else \z@
   \ifnum \@lastchclass = 10 \else
   \edef\@nextchar{\expandafter\string\@nextchar}%
   \@chnum
   \if \@nextchar c\z@ \else
    \if \@nextchar l\@ne \else
     \if \@nextchar r\tw@ \else
   \z@ \@chclass
   \if\@nextchar |\@ne \else
    \if \@nextchar !6 \else
     \if \@nextchar @7 \else
      \if \@nextchar (8 \else
       \if \@nextchar )9 \else
  10
  \@chnum
  \if \@nextchar m\thr@@\else
   \if \@nextchar p4 \else
    \if \@nextchar b5 \else
   \z@ \@chclass \z@ \@preamerr \z@ \fi \fi \fi \fi
   \fi \fi  \fi  \fi  \fi  \fi  \fi \fi \fi \fi \fi \fi}
\gdef\arraybackslash{\let\\=\@arraycr}
\def\@textsubscript#1{{\m@th\ensuremath{_{\mbox{\fontsize\sf@size\z@#1}}}}}
\def\Panel#1#2#3#4{\multicolumn{#3}{){\columncolor{#2}}#4}{#1}}
\def\abbr{}
\def\corr{}
\def\expan{}
\def\gap{}
\def\orig{}
\def\reg{}
\def\ref{}
\def\sic{}
\def\persName{}\def\name{}
\def\placeName{}
\def\orgName{}
\def\textcal#1{{\fontspec{Lucida Calligraphy}#1}}
\def\textgothic#1{{\fontspec{Lucida Blackletter}#1}}
\def\textlarge#1{{\large #1}}
\def\textoverbar#1{\ensuremath{\overline{#1}}}
\def\textquoted#1{‘#1’}
\def\textsmall#1{{\small #1}}
\def\textsubscript#1{\@textsubscript{\selectfont#1}}
\def\textxi{\ensuremath{\xi}}
\def\titlem{\itshape}
\newenvironment{biblfree}{}{\ifvmode\par\fi }
\newenvironment{bibl}{}{}
\newenvironment{byline}{\vskip6pt\itshape\fontsize{16pt}{18pt}\selectfont}{\par }
\newenvironment{citbibl}{}{\ifvmode\par\fi }
\newenvironment{docAuthor}{\ifvmode\vskip4pt\fontsize{16pt}{18pt}\selectfont\fi\itshape}{\ifvmode\par\fi }
\newenvironment{docDate}{}{\ifvmode\par\fi }
\newenvironment{docImprint}{\vskip 6pt}{\ifvmode\par\fi }
\newenvironment{docTitle}{\vskip6pt\bfseries\fontsize{18pt}{22pt}\selectfont}{\par }
\newenvironment{msHead}{\vskip 6pt}{\par}
\newenvironment{msItem}{\vskip 6pt}{\par}
\newenvironment{rubric}{}{}
\newenvironment{titlePart}{}{\par }

\newcolumntype{L}[1]{){\raggedright\arraybackslash}p{#1}}
\newcolumntype{C}[1]{){\centering\arraybackslash}p{#1}}
\newcolumntype{R}[1]{){\raggedleft\arraybackslash}p{#1}}
\newcolumntype{P}[1]{){\arraybackslash}p{#1}}
\newcolumntype{B}[1]{){\arraybackslash}b{#1}}
\newcolumntype{M}[1]{){\arraybackslash}m{#1}}
\definecolor{label}{gray}{0.75}
\def\unusedattribute#1{\sout{\textcolor{label}{#1}}}
\DeclareRobustCommand*{\xref}{\hyper@normalise\xref@}
\def\xref@#1#2{\hyper@linkurl{#2}{#1}}
\begingroup
\catcode`\_=\active
\gdef_#1{\ensuremath{\sb{\mathrm{#1}}}}
\endgroup
\mathcode`\_=\string"8000
\catcode`\_=12\relax

\usepackage[a4paper,twoside,lmargin=1in,rmargin=1in,tmargin=1in,bmargin=1in,marginparwidth=0.75in]{geometry}
\usepackage{framed}

\definecolor{shadecolor}{gray}{0.95}
\usepackage{longtable}
\usepackage[normalem]{ulem}
\usepackage{fancyvrb}
\usepackage{fancyhdr}
\usepackage{graphicx}
\usepackage{marginnote}

\renewcommand*{\marginfont}{\itshape\footnotesize}

\def\Gin@extensions{.pdf,.png,.jpg,.mps,.tif}

  \pagestyle{fancy}

\usepackage[pdftitle={AM 62 fol.},
 pdfauthor={}]{hyperref}
\hyperbaseurl{}

	 \paperwidth210mm
	 \paperheight297mm
              
\def\@pnumwidth{1.55em}
\def\@tocrmarg {2.55em}
\def\@dotsep{4.5}
\setcounter{tocdepth}{3}
\clubpenalty=8000
\emergencystretch 3em
\hbadness=4000
\hyphenpenalty=400
\pretolerance=750
\tolerance=2000
\vbadness=4000
\widowpenalty=10000

\renewcommand\section{\@startsection {section}{1}{\z@}%
     {-1.75ex \@plus -0.5ex \@minus -.2ex}%
     {0.5ex \@plus .2ex}%
     {\reset@font\Large\bfseries\sffamily}}
\renewcommand\subsection{\@startsection{subsection}{2}{\z@}%
     {-1.75ex\@plus -0.5ex \@minus- .2ex}%
     {0.5ex \@plus .2ex}%
     {\reset@font\Large\sffamily}}
\renewcommand\subsubsection{\@startsection{subsubsection}{3}{\z@}%
     {-1.5ex\@plus -0.35ex \@minus -.2ex}%
     {0.5ex \@plus .2ex}%
     {\reset@font\large\sffamily}}
\renewcommand\paragraph{\@startsection{paragraph}{4}{\z@}%
     {-1ex \@plus-0.35ex \@minus -0.2ex}%
     {0.5ex \@plus .2ex}%
     {\reset@font\normalsize\sffamily}}
\renewcommand\subparagraph{\@startsection{subparagraph}{5}{\parindent}%
     {1.5ex \@plus1ex \@minus .2ex}%
     {-1em}%
     {\reset@font\normalsize\bfseries}}


\def\l@section#1#2{\addpenalty{\@secpenalty} \addvspace{1.0em plus 1pt}
 \@tempdima 1.5em \begingroup
 \parindent \z@ \rightskip \@pnumwidth 
 \parfillskip -\@pnumwidth 
 \bfseries \leavevmode #1\hfil \hbox to\@pnumwidth{\hss #2}\par
 \endgroup}
\def\l@subsection{\@dottedtocline{2}{1.5em}{2.3em}}
\def\l@subsubsection{\@dottedtocline{3}{3.8em}{3.2em}}
\def\l@paragraph{\@dottedtocline{4}{7.0em}{4.1em}}
\def\l@subparagraph{\@dottedtocline{5}{10em}{5em}}
\@ifundefined{c@section}{\newcounter{section}}{}
\@ifundefined{c@chapter}{\newcounter{chapter}}{}
\newif\if@mainmatter 
\@mainmattertrue
\def\chaptername{Chapter}
\def\frontmatter{%
  \pagenumbering{roman}
  \def\thechapter{\@roman\c@chapter}
  \def\theHchapter{\roman{chapter}}
  \def\@chapapp{}%
}
\def\mainmatter{%
  \cleardoublepage
  \def\thechapter{\@arabic\c@chapter}
  \setcounter{chapter}{0}
  \setcounter{section}{0}
  \pagenumbering{arabic}
  \setcounter{secnumdepth}{6}
  \def\@chapapp{\chaptername}%
  \def\theHchapter{\arabic{chapter}}
}
\def\backmatter{%
  \cleardoublepage
  \setcounter{chapter}{0}
  \setcounter{section}{0}
  \setcounter{secnumdepth}{2}
  \def\@chapapp{\appendixname}%
  \def\thechapter{\@Alph\c@chapter}
  \def\theHchapter{\Alph{chapter}}
  \appendix
}
\newenvironment{bibitemlist}[1]{%
   \list{\@biblabel{\@arabic\c@enumiv}}%
       {\settowidth\labelwidth{\@biblabel{#1}}%
        \leftmargin\labelwidth
        \advance\leftmargin\labelsep
        \@openbib@code
        \usecounter{enumiv}%
        \let\p@enumiv\@empty
        \renewcommand\theenumiv{\@arabic\c@enumiv}%
	}%
  \sloppy
  \clubpenalty4000
  \@clubpenalty \clubpenalty
  \widowpenalty4000%
  \sfcode`\.\@m}%
  {\def\@noitemerr
    {\@latex@warning{Empty `bibitemlist' environment}}%
    \endlist}

\def\tableofcontents{\section*{\contentsname}\@starttoc{toc}}
\parskip0pt
\parindent1em
\def\Panel#1#2#3#4{\multicolumn{#3}{){\columncolor{#2}}#4}{#1}}
\newenvironment{reflist}{%
  \begin{raggedright}\begin{list}{}
  {%
   \setlength{\topsep}{0pt}%
   \setlength{\rightmargin}{0.25in}%
   \setlength{\itemsep}{0pt}%
   \setlength{\itemindent}{0pt}%
   \setlength{\parskip}{0pt}%
   \setlength{\parsep}{2pt}%
   \def\makelabel##1{\itshape ##1}}%
  }
  {\end{list}\end{raggedright}}
\newenvironment{sansreflist}{%
  \begin{raggedright}\begin{list}{}
  {%
   \setlength{\topsep}{0pt}%
   \setlength{\rightmargin}{0.25in}%
   \setlength{\itemindent}{0pt}%
   \setlength{\parskip}{0pt}%
   \setlength{\itemsep}{0pt}%
   \setlength{\parsep}{2pt}%
   \def\makelabel##1{\upshape\sffamily ##1}}%
  }
  {\end{list}\end{raggedright}}
\newenvironment{specHead}[2]%
 {\vspace{20pt}\hrule\vspace{10pt}%
  \label{#1}\markright{#2}%

  \pdfbookmark[2]{#2}{#1}%
  \hspace{-0.75in}{\bfseries\fontsize{16pt}{18pt}\selectfont#2}%
  }{}
      \def\TheFullDate{1970-01-01}
\def\TheID{\makeatother }
\def\TheDate{1970-01-01}
\title{AM 62 fol.}
\author{}\makeatletter 
\makeatletter
\newcommand*{\cleartoleftpage}{%
  \clearpage
    \if@twoside
    \ifodd\c@page
      \hbox{}\newpage
      \if@twocolumn
        \hbox{}\newpage
      \fi
    \fi
  \fi
}
\makeatother
\makeatletter
\thispagestyle{empty}
\markright{\@title}\markboth{\@title}{\@author}
\renewcommand\small{\@setfontsize\small{9pt}{11pt}\abovedisplayskip 8.5\p@ plus3\p@ minus4\p@
\belowdisplayskip \abovedisplayskip
\abovedisplayshortskip \z@ plus2\p@
\belowdisplayshortskip 4\p@ plus2\p@ minus2\p@
\def\@listi{\leftmargin\leftmargini
               \topsep 2\p@ plus1\p@ minus1\p@
               \parsep 2\p@ plus\p@ minus\p@
               \itemsep 1pt}
}
\makeatother
\fvset{frame=single,numberblanklines=false,xleftmargin=5mm,xrightmargin=5mm}
\fancyhf{} 
\setlength{\headheight}{14pt}
\fancyhead[LE]{\bfseries\leftmark} 
\fancyhead[RO]{\bfseries\rightmark} 
\fancyfoot[RO]{}
\fancyfoot[CO]{\thepage}
\fancyfoot[LO]{\TheID}
\fancyfoot[LE]{}
\fancyfoot[CE]{\thepage}
\fancyfoot[RE]{\TheID}
\hypersetup{linkbordercolor=0.75 0.75 0.75,urlbordercolor=0.75 0.75 0.75,bookmarksnumbered=true}
\fancypagestyle{plain}{\fancyhead{}\renewcommand{\headrulewidth}{0pt}}\makeatother 
\begin{document}

\makeatletter
\noindent\parbox[b]{.75\textwidth}{\fontsize{14pt}{16pt}\bfseries\raggedright\sffamily\selectfont \@title}
\vskip20pt
\par\noindent{\fontsize{11pt}{13pt}\sffamily\itshape\raggedright\selectfont\@author\hfill\TheDate}
\vspace{18pt}
\makeatother
\let\tabcellsep& \begin{bibitemlist}{1}
\bibitem[Amsler and Tompa (1988)]{DI-BIBL-1}\label{DI-BIBL-1}Robert A. Amsler, Frank W. Tompa. ‘An SGML-Based Standard for English Monolingual Dictionaries’. \textit{Information in Text}, \textit{Fourth Annual Conference of the U[niversity of] W[aterloo] Centre for the New Oxford English Dictionary},  (Fourth Annual Conference of the U[niversity of] W[aterloo] Centre for the New Oxford English Dictionary, October 26-28, 1988, Waterloo, Canada) October 1988. Waterloo, Canada. pp. 61-79. 
\bibitem[Anglo-American Cataloguing Rules (2002–2005)]{HD-BIBL-1}\label{HD-BIBL-1}\textit{Anglo-American Cataloguing Rules}, Second Edition, 2002 revision, 2005 update. 2002–2005. Chicago, Ottawa: American Library Association. Canadian Library Association. 
\bibitem[National Information Standards Organization (2010)]{ANSI-NISO-Z39.29}\label{ANSI-NISO-Z39.29}National Information Standards Organization. \textit{ANSI/NISO Z39.29 – 2005 (R2010) Bibliographic References}, 2010. 
\bibitem[Berglund et al. (eds.) (2007)]{XPATH2}\label{XPATH2}Anders Berglund, Scot Boag, Mary F. Fernández, Michael Kay, Jonathan Robie, Jérôme Siméon (eds.) \textit{XML Path Language (XPath) 2.0}, 23 January 2007. W3C.  <\url{http://www.w3.org/TR/xpath20/}>.
\bibitem[Berglund (ed.) (2006)]{XSL11}\label{XSL11}Anders Berglund (ed.) \textit{Extensible Stylesheet Language (XSL) Version 1.1}, 5 December 2006. W3C.  <\url{http://www.w3.org/TR/xsl11}>.
\bibitem[Bray et al. (eds.) (2006)]{XMLREC}\label{XMLREC}Tim Bray, Jean Paoli, C. M. Sperberg-McQueen, Eve Maler, François Yergau (eds.) \textit{Extensible Markup Language (XML) Version 1.0 (Fourth edition)}, 16 August 2006. W3C.  <\url{http://www.w3.org/TR/REC-xml/}>.
\bibitem[Bray et al. (eds.) (2006)]{NAMESPACES}\label{NAMESPACES}Tim Bray, Dave Hollander, Andrew Laymon, Richard Tobin (eds.) \textit{Namespaces in XML 1.0 (second edition)}, 16 August 2006. W3C.  <\url{http://www.w3.org/TR/xml-names/}>.
\bibitem[Klinkenborg and Cahoon (1981)]{KLINKENBORG}\label{KLINKENBORG}\textit{British Literary Manuscripts. Series 2: from 1800 to 1914}, Verlyn Klinkenborg, Herbert Cahoon. 1981. New York: Pierpont Morgan Library. 
\bibitem[British Standards Institute (1990)]{BS-5605}\label{BS-5605}British Standards Institute. \textit{BS 5605:1990: Recommendations for Citing and Referencing Published Material}, 1990. 
\bibitem[British Standards Institute (1983)]{BS-6371}\label{BS-6371}British Standards Institute. \textit{BS 6371:1983: Recommendations for Citation of Unpublished Documents}, 1983. 
\bibitem[Burnard (1988)]{AB-eg-01}\label{AB-eg-01}Lou Burnard. ‘Report of Workshop on Text Encoding Guidelines’. \textit{Literary \& Linguistic Computing} 1988. 3. 
\bibitem[Burnard and Sperberg-McQueen (1995)]{Burnard1995b}\label{Burnard1995b}Lou Burnard, C. Michael Sperberg-McQueen. ‘The Design of the TEI Encoding Scheme’. \textit{Computers and the Humanities} 1995. 29  (1)  p. 17–39.  http://dx.doi.org/10.1007/BF01830314 (Reprinted in Ide1995b, pp. 17-40)
\bibitem[Burnard and Rahtz (2004)]{TD-BIBL-01}\label{TD-BIBL-01}Lou Burnard, Sebastian Rahtz. \textit{RelaxNG with Son of ODD}, \textit{Proceedings of Extreme Markup Languages 2004}, 2004. \url{http://www.mulberrytech.com/Extreme/Proceedings/html/2004/Burnard01/EML2004Burnard01.pdf}
\bibitem[Burrows (1987)]{HD-BIBL-2}\label{HD-BIBL-2}John Burrows. \textit{Computation into Criticism: A Study of Jane Austen's Novel and an Experiment in Method}, 1987. Oxford: Clarendon Press. 
\bibitem[Calzolari et al. (1990)]{DI-BIBL-2}\label{DI-BIBL-2}N. Calzolari, C. Peters, A. Roventini. \textit{Computational Model of the Dictionary Entry: Preliminary Report}, \textit{Acquilex: Esprit Basic Research Action No. 3030, Six-Month Deliverable}, April 1990. Pisa. 
\bibitem[Carlisle et al. (eds.) (2003)]{MATHML}\label{MATHML}David Carlisle, Patrick Ion, Robert Miner, Nico Poppelier (eds.) \textit{Mathematical Markup Language (MathML) Version 2.0 (Second edition)}, 21 October 2003. W3C.  <\url{http://www.w3.org/TR/MathML2/}>.
\bibitem[Carpenter (1992)]{FS-BIBL-5}\label{FS-BIBL-5}Bob Carpenter. \textit{The logic of typed feature structures}, 1992. Cambridge: Cambridge University Press. Cambridge Tracts in Theoretical Computer Science 32. 
\bibitem[Chartrand and Lesniak (1986)]{GD-BIBL-1}\label{GD-BIBL-1}Gary Chartrand, Linda Lesniak. \textit{Graphs and Digraphs}, 1986. Menlo Park, CA: Wadsworth. 
\bibitem[Chatti et al. (2007)]{NH-BIBL-5}\label{NH-BIBL-5}Noureddine Chatti, Suha Kaouk, Sylvie Calabretto, Jean Marie Pinon. ‘MultiX: an XML based formalism to encode multistructured documents’. \textit{Proceedings of Extreme Markup Languages 2007}, 2007.  <\url{http://www.idealliance.org/papers/extreme/proceedings/html/2007/Chatti01/EML2007Chatti01.html}>.
\bibitem[Clark (ed.) (1999)]{XSLT}\label{XSLT}James Clark (ed.) \textit{XSL Transformations (XSLT) Version 1.0}, 16 November 1999. W3C.  <\url{http://www.w3.org/TR/xslt/}>.
\bibitem[Clark and DeRose (eds.) (1999)]{XPATH}\label{XPATH}James Clark, Steve DeRose (eds.) \textit{XML Path Language (XPath) Version 1.0}, 16 November 1999. W3C.  <\url{http://www.w3.org/TR/xpath/}>.
\bibitem[DANLEX Group (1987)]{DI-BIBL-7}\label{DI-BIBL-7}The DANLEX Group. ‘Descriptive tools for electronic processing of dictionary data’. \textit{Lexicographica, Series Maior} 1987. Tübingen: Niemeyer. 
\bibitem[Davis et al. (2006)]{WD-bibl-01}\label{WD-bibl-01}Mark Davis, Ken Whistler, Asmus Freytag. \textit{Unicode Character Database}, 2006. Unicode Consortium.  <\url{http://www.unicode.org/Public/UNIDATA/UCD.html}>.
\bibitem[Dekhtyar and Iacob (2005)]{NH-BIBL-3}\label{NH-BIBL-3}Alex Dekhtyar, Ionut E. Iacob. \textit{A framework for management of concurrent XML markup}, 2005.  <\url{http://www.eppt.org/\textasciitilde emil/publications/dke04-concurrent.pdf}>.
\bibitem[DeRose (2004)]{NH-BIBL-1}\label{NH-BIBL-1}Steven DeRose. ‘Markup overlap: a review and a horse’. \textit{Proceedings of Extreme Markup Languages 2004}, 2004.  <\url{http://www.mulberrytech.com/Extreme/Proceedings/html/2004/DeRose01/EML2004DeRose01.html}>.
\bibitem[Deutsches Institut für Normung (1984)]{DIN-1505-2}\label{DIN-1505-2}Deutsches Institut für Normung. \textit{DIN 1505-2: Titelangaben von Dokumenten; Zitierregeln}, 1984. 
\bibitem[Durusau and O'Donnell (2002)]{NH-BIBL-6}\label{NH-BIBL-6}Patrick Durusau, Matthew Brook O'Donnell. ‘Coming down from the trees: next step in the evolution of markup?’. \textit{Proceedings of Extreme Markup Languages 2002}, 2002. 
\bibitem[Edwards and Lampert (eds.) (1993)]{TS-BIBL-1}\label{TS-BIBL-1}J. A. Edwards, M. D. Lampert (eds.) \textit{Talking Language: Transcription and Coding of Spoken Discourse}, 1993. Hillsdale, N.J.: Lawrence Erlbaum Associates. 
\bibitem[Fought and Van Ess-Dykema]{DI-BIBL-3}\label{DI-BIBL-3}John Fought, Carol Van Ess-Dykema. \textit{Toward an SGML Document Type Definition for Bilingual Dictionaries}, \textit{TEI working paper TEI AIW20}, available from the TEI.. 
\bibitem[Freytag (2006)]{CH-eg-02}\label{CH-eg-02}Asmus Freytag. \textit{The Unicode Character Property Model}, Unicode Technical Report \#232006.  <\url{http://www.unicode.org/reports/tr23/}>.
\bibitem[Gale and Church (1993)]{SA-BIBL-1}\label{SA-BIBL-1}William A. Gale, Kenneth W. Church. ‘Program for aligning sentences in bilingual corpora’. \textit{Computational Linguistics} 1993. 19 pp. 75-102. 
\bibitem[Garside et al. (1991)]{AI-BIBL-7}\label{AI-BIBL-7}R. G. Garside, G. N. Leech, G. R. Sampson. \textit{The Computational Analysis of English: a Corpus-Based Approach}, 1991. Oxford: Oxford University Press. 
\bibitem[Grosso et al. (eds.) (2003)]{XPTRFMWK}\label{XPTRFMWK}Paul Grosso, Eve Maler, Jonathan Marsh, Norman Walsh (eds.) \textit{XPointer Framework}, 25 March 2003. W3C.  <\url{http://www.w3.org/TR/xptr-framework/}>.
\bibitem[Grosso et al. (eds.) (2003)]{XPTRELEM}\label{XPTRELEM}Paul Grosso, Eve Maler, Jonathan Marsh, Norman Walsh (eds.) \textit{XPointer element() Scheme}, 25 March 2003. W3C.  <\url{http://www.w3.org/TR/xptr-element/}>.
\bibitem[Hilbert et al. (2005)]{NH-BIBL-2}\label{NH-BIBL-2}Mirco Hilbert, Oliver Schonefeld, Andreas Witt. ‘Making CONCUR work’. \textit{Proceedings of Extreme Markup Languages 2005}, 2005.  <\url{http://www.mulberrytech.com/Extreme/Proceedings/html/2005/Witt01/EML2005Witt01.xml}>.
\bibitem[Huitfeldt and Sperberg-McQueen (2001)]{NH-BIBL-8}\label{NH-BIBL-8}Claus Huitfeldt, C. Michael Sperberg-McQueen. \textit{TexMECS: An experimental markup meta-language for complex documents}, 2001.  <\url{http://decentius.aksis.uib.no/mlcd/2003/Papers/texmecs.html}>.
\bibitem[Ide et al. (1992)]{DI-BIBL-6}\label{DI-BIBL-6}Nancy Ide, Jean Veronis, Susan Warwick-Amstrong, Nicoletta Calzolari. ‘Principles for Encoding machine readable dictionaries’. \textit{Proceedings of the Fifth EURALEX International Congress, EURALEX'92},  (Fifth EURALEX International Congress, EURALEX'92, University of Tampere, Finland) 1992. 
\bibitem[Ide et al. (1993)]{DI-BIBL-5}\label{DI-BIBL-5}Nancy Ide, Jacques Le Maitre, Jean Veronis. ‘Outline of a Model for Lexical Databases’. \textit{Information Processing and Management} 1993. 29  (2)  pp. 159-186. 
\bibitem[Ide and Veronis (1995)]{DI-BIBL-4}\label{DI-BIBL-4}Nancy Ide, Jean Veronis. ‘Encoding Print Dictionaries’. \textit{Computers and the Humanities} 1995. 29 pp. 167-195. 
\bibitem[Ide et al. (2000)]{DI-BIBL-9}\label{DI-BIBL-9}N. Ide, A. Kilgarriff, L. Romary. ‘A Formal Model of Dictionary Structure and Content’. \textit{Proceedings of Euralex 2000},  (Euralex 2000) 2000. Stuttgart. pp. 113-126. 
\bibitem[Ide and Romary (2004)]{IdeRomary}\label{IdeRomary}Nancy Ide, Laurent Romary. ‘A Registry of Standard Data Categories for Linguistic Annotation’. \textit{Proceedings of the 4th International Conference on Language Resources and Evaluation - LREC'04}, 2004. Lisbon, Portugal. pp. 135-138.  {\ref http://hal.inria.fr/inria-00099858} (Describes the development of a Data Category Registry (DCR) component for the Linguistic Annotation Framework; an ISO standard of major importance in the encoding of linguistic analysis. )
\bibitem[ISBD: International Standard Bibliographic Description IFLA Series on Bibliographic Control (2011)]{ISBD}\label{ISBD}\textit{ISBD: International Standard Bibliographic Description}, 2011. Berlin, München, De Gruyter Saur. \textit{IFLA Series on Bibliographic Control},  44. 
\bibitem[International Organization for Standardization (2009)]{ISO-12620}\label{ISO-12620}International Organization for Standardization. \textit{ISO 12620:2009: Terminology and other language and content resources – Specification of data categories and management of a Data Category Registry for language resources}, 2009. 
\bibitem[International Organization for Standardization (1987)]{ISO-690}\label{ISO-690}International Organization for Standardization. \textit{ISO 690:1987: Information and documentation – Bibliographic references – Content, form and structure}, 1987. 
\bibitem[Jackendoff (1977)]{GD-BIBL-2}\label{GD-BIBL-2}R. Jackendoff. ‘X-Bar Syntax: A study of phrase structure’. \textit{Linguistic Inquiry Monograph} 1977. 2. 
\bibitem[Jagadish et al. (2004)]{NH-BIBL-4}\label{NH-BIBL-4}H. V. Jagadish, Laks V. S. Lakshmanan, Monica Scannapieco, Divesh Srivastava, Nuwee Wiwatwattana. \textit{Colorful XML: one hierarchy isn't enough}, 2004.  <\url{http://www.research.att.com/\textasciitilde divesh/papers/jlssw2004-mct.pdf}>.
\bibitem[Johansson et al. (1991)]{TS-BIBL-3}\label{TS-BIBL-3}Stig Johansson, Lou Burnard, Jane Edwards, And Rosta. \textit{Working Paper on Spoken Texts}, \textit{TEI document TEI AI2 W1}, 1991. 
\bibitem[Johansson (1994)]{TS-BIBL-2}\label{TS-BIBL-2}Stig Johansson. ‘Encoding a Corpus in Machine-Readable Form’. Sue Atkins, Antonio Zampolli (eds.) \textit{Computational Approaches to the Lexicon: An Overview}, 1994. Oxford: Oxford University Press. 
\bibitem[Kay (ed.) (2007)]{XSLT2}\label{XSLT2}Michael Kay (ed.) \textit{XSL Transformations (XSLT) Version 2.0}, 23 January 2007. W3C.  <\url{http://www.w3.org/TR/xslt20/}>.
\bibitem[Knuth (1992)]{KNUTH}\label{KNUTH}Donald E. Knuth. Literate Programming, CSLI Lecture Notes 271992. Stanford, California: Center for the Study of Language and Information.  0-937073-80-6 
\bibitem[Kytö and Rissanen (1988)]{CC-BIBL-1}\label{CC-BIBL-1}M. Kytö, M. Rissanen. ‘The Helsinki Corpus of English Texts’. M. Kytö, O. Ihalainen, M. Rissanen (eds.) \textit{Corpus Linguistics: hard and soft}, 1988. Amsterdam: Rodopi. 
\bibitem[Langendoen and Simons (1995)]{FS-BIBL-01}\label{FS-BIBL-01}D. Terence Langendoen, Gary F.  Simons. ‘A rationale for the TEI recommendations for feature-structure markup,’. \textit{Computers and the Humanities} 1995. 29 pp. 167-195. 
\bibitem[Leech and Garside (1991)]{AI-BIBL-5}\label{AI-BIBL-5}G. N. Leech, R. G. Garside. ‘Running a Grammar Factory’. S. Johansson, A.-B. Stenstrøm (eds.) \textit{English Computer Corpora: Selected Papers and Research Guide}, 1991. Berlin, New York: de Gruyter. Mouton. pp. pp. 15-32.. 
\bibitem[Lie and Bos (eds.) (1999)]{CSS1}\label{CSS1}Håkon Wium Lie, Bert Bos (eds.) \textit{Cascading Style Sheets, Level 1}, 11 January 1999. W3C.  <\url{http://www.w3.org/TR/REC-CSS1/}>.
\bibitem[Loman and Jørgensen (1971)]{TS-BIBL-7}\label{TS-BIBL-7}Bengt Loman, Nils Jørgensen. \textit{Manual for analys och beskrivning av makrosyntagmer}, 1971. Lund: Studentlitteratur. 
\bibitem[MacWhinney (1988)]{TS-BIBL-4}\label{TS-BIBL-4}Brian MacWhinney. \textit{CHAT Manual}, 1988. Pittsburgh: Dept of Psychology, Carnegie-Mellon University. pp. 87ff. 
\bibitem[Marsh (ed.) (2001)]{XMLBASE}\label{XMLBASE}Jonathan Marsh (ed.) \textit{XML Base}, 27 June 2001. W3C.  <\url{http://www.w3.org/TR/xmlbase/}>.
\bibitem[Marshall (1983)]{AI-BIBL-6}\label{AI-BIBL-6}I. Marshall. ‘Choice of Grammatical Word Class without Global Syntactic Analysis: Tagging Words in the LOB Corpus’. \textit{Computers and the Humanities} 1983. 17 pp. 139-50. 
\bibitem[Mattheier et al. (eds.) (1988)]{CO-BIBL-1}\label{CO-BIBL-1}Klaus Mattheier, Ulrich Ammon, Peter Trudgill (eds.) \textit{Sociolinguistics}, \textit{Soziolinguistik}, \textit{An international handbook of the science of language and society}, \textit{Ein internationales Handbuch zur Wissenschaft von Sprache und Gesellschaft}, 1988. Berlin, New York: De Gruyter. I pp. 271 and 274. 
\bibitem[Parkes (1969)]{PARKES}\label{PARKES}M. B. Parkes. \textit{English Cursive Book Hands 1250–1500}, 1969. Oxford: Clarendon Press. 
\bibitem[Pereira (1987)]{FS-BIBL-1}\label{FS-BIBL-1}Fernando C. N. Pereira. \textit{Grammars and logics of partial information}, 1987. Menlo Park, CA: SRI International. SRI International Technical Note 420. 
\bibitem[Petty (1977)]{PETTY}\label{PETTY}A. G. Petty. \textit{English literary hands from Chaucer to Dryden}, 1977. London: Edward Arnold. pp. 22–25. 
\bibitem[Phillips and Davis (eds.) (2006)]{CH-BIBL-4}\label{CH-BIBL-4}Addison Phillips, Mark Davis (eds.) \textit{Tags for Identifying Languages}, 2006. IETF.  RFC 4646
\bibitem[Phillips and Davis (eds.) (2006)]{CH-BIBL-5}\label{CH-BIBL-5}Addison Phillips, Mark Davis (eds.) \textit{Matching of Language Tags}, 2006. IETF.  RFC 4647
\bibitem[Ragget et al. (eds.) (1999)]{HTML4}\label{HTML4}Dave Ragget, Arnaud Le Hors, Ian Jacobs (eds.) \textit{HTML 4.01 Specification}, 24 December 1999. W3C.  <\url{http://www.w3.org/TR/html401/}>.
\bibitem[Die Deutsche Bibliothek (2006)]{RAK}\label{RAK}Die Deutsche Bibliothek. \textit{Regeln für die alphabetische Katalogisierung in wissenschaftlichen Bibliotheken RAK-WB}, 2006. 
\bibitem[Istituto Centrale per il Catalogo Unico (1979)]{RICA}\label{RICA}Istituto Centrale per il Catalogo Unico. \textit{Regole italiane di catalogazione per autori}, 1979. 
\bibitem[Renear et al. (1996)]{SG-BIBL-2}\label{SG-BIBL-2}A. Renear, E. Mylonas, D.  Durand. ‘Refining our notion of what text really is: the problem of overlapping hierarchies’. Nancy Ide, Susan Hockey (eds.) \textit{Research in Humanities Computing}, 1996. Oxford University Press. 
\bibitem[Shieber (1986)]{FS-BIBL-2}\label{FS-BIBL-2}Stuart Shieber. \textit{An Introduction to Unification-based Approaches to Grammar}, 1986. Palo Alto, CA: Center for the Study of Language and Information.  CSLI Lecture Notes 4
\bibitem[Tennison and Piez (2002)]{NH-BIBL-7}\label{NH-BIBL-7}Jeni Tennison, Wendell Piez. ‘The layered markup and annotation language’. \textit{Proceedings of Extreme Markup Languages Conference}, 2002. 
\bibitem[Unicode Consortium (2006)]{CH-BIBL-3}\label{CH-BIBL-3}\textit{The Unicode Standard, Version 5.0}, Unicode Consortium. 2006. Addison-Wesley Professional.  <\url{http://www.unicode.org/}>.
\bibitem[Tutin and Veronis (1998)]{DI-BIBL-8}\label{DI-BIBL-8}Agnès Tutin, Jean Veronis. ‘Electronic dictionary encoding: customizing the TEI Guidelines’. \textit{Proceedings of the Eighth Euralex International Congress},  (Eighth Euralex International Congress) 1998. 
\bibitem[van der Vlist (2004)]{SG-BIBL-1}\label{SG-BIBL-1}Eric van der Vlist. \textit{RELAX NG}, 2004. O'Reilly. 
\bibitem[Witt (2002)]{NH-BIBL-01}\label{NH-BIBL-01}Andreas Witt. \textit{Multiple Informationsstrukturierung mit Auszeichnungssprachen. XML-basierte Methoden und deren Nutzen für die Sprachtechnologie}, 2002.  (Ph D thesis, Bielefeld University) (See also \url{http://xml.coverpages.org/Witt-allc2002.html})
\bibitem[XHTML™ 1.0 The Extensible HyperText Markup Language (Second
Edition) (2000)]{XHTML}\label{XHTML}\textit{XHTML™ 1.0 The Extensible HyperText Markup Language (Second Edition)}, 26 January 2000. W3C.  <\url{http://www.w3.org/TR/xhtml/}>.
\end{bibitemlist}

\section{Identification}
Denmark, København, Den Arnamagnæanske Samling, AM 62 fol. 74
\section[{Ólafs saga Tryggvasonar en mesta; 							Iceland, s. XIV 							2/2. This version of Ólafs saga Tryggvasonar en 								mesta incorporates various sagas or excerps of sagas and þættir about 							Ólafur, also at the point, where texts from other redactions of the saga end (here: 							fol. 51va:24 til englandz af syrlandi).}]{\textit{Ólafs saga Tryggvasonar en mesta}; Iceland, s. XIV 2/2.  (This version of \textit{Ólafs saga Tryggvasonar en mesta} incorporates various sagas or excerps of sagas and þættir about Ólafur, also at the point, where texts from other redactions of the saga end (here: fol. 51va:24 ‘til englandz af syrlandi’).)}\label{AM02-062-en}
\section{Contents}
\begin{msItem} \textbf{1ra-51va:24}:  \textit{Ólafs saga Tryggvasonar en mesta}, \textit{Language of text}: Old Norse/Icelandic
\subsection{Incipit}
hann gret þo eigi. ok er Þorolfr vard þessa varr letti hann vpp
\subsection{Explicit}
er {\hskip1pt}\\{} þa voro nykomnir til englandz af syrlandi \begin{bibitemlist}{1}
 \bibitem {bibitem-1}\xref{FMS1}{Fornmanna sögur}I 71-306  (Ed. S);II  (Ed. S); III 1-64 (Ed. S)
 \bibitem {bibitem-2}Ólafur Halldórsson, \xref{EA-A1}{Óláfs saga Tryggvasonar en mesta} I  73:8-400   (Ed. D\textsuperscript{1}. The text starts in the footnotes.);II  (Ed. D\textsuperscript{1})
 \bibitem {bibitem-3}Bjarni Einarsson, \xref{Rit15}{Hallfreðar saga 1977} 3-21  (Fols 18va:13-19rb:10)
 \bibitem {bibitem-4}W. van Eeden, \xref{Hallfr1919}{Hallfreðar saga 1919} 109-112 (Text A. Fols 18va:13-19va:16); 113 (Text B. Fols 22ra:33-b:11); 113-118 (Text C. Fols 24va:12-25vb:32); 118-123 (Text DE. Fols 28ra:9-29ra); 123-126 (Text G. Fols 48ra-va:34)

\end{bibitemlist}
\end{msItem} \begin{msItem} \textbf{29rb-31va:23}:  \textit{Norna-Gests þáttr}
\subsection{Incipit}
ÞAt var a eini nott at
\subsection{Explicit}
ok þottí sannaz vm lifdaga hans sva sem hann sagdí\begin{bibitemlist}{1}
 \bibitem {bibitem-5}Bugge, \xref{NSaSI6}{Söguþáttr af Norna-Gesti}
 \bibitem {bibitem-6}Ólafur Halldórsson, \xref{EA-A3}{Óláfs saga Tryggvasonar en mesta} III  15-38  (Ed. D\textsuperscript{1})
 \bibitem {bibitem-7}Cipolla, \xref{NornaGestr1996}{Il racconto di Nornagestr} 227-244

\end{bibitemlist}
\end{msItem} \begin{msItem} \textbf{31va:23-32rb}:  \textit{Helga Þáttr Þórissonar}  {\rubric her segir fra {\hskip1pt}\\{} godmun {\hskip1pt}\\{}di kkonvngi {\hskip1pt}\\{} af gl {\hskip1pt}\\{}asisu{\hskip1pt}\\{}ollum}
\subsection{Incipit}
Þorir het madr er bío j noregi
\subsection{Explicit}
ok hefir engí {\hskip1pt}\\{} madr þav sed sidanok lykr her þessi sǫgv etc.\begin{biblfree} Ólafur Halldórsson, \xref{EA-A3}{Óláfs saga Tryggvasonar en mesta} III  38-44  (Ed. D\textsuperscript{1})\end{biblfree} \end{msItem} \begin{msItem} \textbf{33va-37vb}:  \textit{Færeyinga saga}  {\rubric her hefir færeyíngar þattr}
\subsection{Incipit}
Madr er nefndr grimr kamban
\subsection{Explicit}
varist [?] segía henni hvert þat [...] {\gap }dit\begin{biblfree} Ólafur Halldórsson, \xref{Rit30}{Færeyinga saga} 3-5, 8-11, 13-37, 39-41, 43, 49-67  (Bottom text. Ed. D)\end{biblfree} \end{msItem} \begin{msItem} \textbf{44rb:3-46ra}:  \textit{Óláfr konungr braut goð þrænda}  {\rubric Capitulum}
\subsection{Incipit}
⟨E⟩n er olafr konvngr hafdi skama stvnd verít i þrandheimi
\subsection{Explicit}
ef þer hafít adr illa ok oheyrilí{\hskip1pt}\\{}ga af honvm gengít\begin{biblfree} Ólafur Halldórsson, \xref{EA-A3}{Óláfs saga Tryggvasonar en mesta} III  1-14  (Ed. D\textsuperscript{1})\end{biblfree} \end{msItem} \begin{msItem} \textbf{50ra:2-b:4}:  \textit{Tryggva Óláfssonar hefnt}  {\rubric Capitulum}
\subsection{Incipit}
Sva bar at eítt sínn at haralldr konvngr {\hskip1pt}\\{} kom þar
\subsection{Explicit}
ok hafdi hann med ser j hírd sínní\begin{biblfree} Ólafur Halldórsson, \xref{EA-A3}{Óláfs saga Tryggvasonar en mesta} III  45-47  (Ed. D\textsuperscript{1})\end{biblfree} \end{msItem} \begin{msItem} \textbf{51va:25-52va:36}:  \textit{Halldórs Þáttr Snorrasonar}  {\rubric sidasti þattr olaf{\hskip1pt}\\{}ssaug{\hskip1pt}\\{}o try{\hskip1pt}\\{}ggua{\hskip1pt}\\{}son⟨ar⟩ {\hskip1pt}\\{} noreg{\hskip1pt}\\{}s konvngs}
\subsection{Incipit}
Halldor son snora goda af islandi var med haraldi konvngí sigvrdar syni
\subsection{Explicit}
hann hafdi drepit {\hskip1pt}\\{} hirdmann haralldz konvngs ok hafdi ⟨hann⟩ þvi reidi a honvm\begin{bibitemlist}{1}
 \bibitem {bibitem-12}\xref{FMS3}{Fornmanna sögur} III  152-163:2  (Ed. S)
 \bibitem {bibitem-13}Ólafur Halldórsson, \xref{EA-A3}{Óláfs saga Tryggvasonar en mesta} III  47-57  (Ed. D\textsuperscript{1})

\end{bibitemlist}
\end{msItem} \begin{msItem} \textbf{52vb:10-53vb:10}:  \textit{Frá Sigurði byskupi}
\subsection{Incipit}
Sęmilígr kenní madr ok godrar mínníngar
\subsection{Explicit}
voro þav sammędd syskin in heilagri {\hskip1pt}\\{} olafr konvngr ok fyrnefnd [ {\corr gvnnhilldr}].\begin{bibitemlist}{1}
 \bibitem {bibitem-14}\xref{FMS3}{Fornmanna sögur} III  163:4-172  (Ed. S)
 \bibitem {bibitem-15}Ólafur Halldórsson, \xref{EA-A3}{Óláfs saga Tryggvasonar en mesta} III  57-64  (Ed. D\textsuperscript{1})

\end{bibitemlist}
\end{msItem} \begin{msItem} \textbf{53vb:14-31}:  \textit{Sýn Brestis}  {\rubric Kapitulum [...] {\gap }}
\subsection{Incipit}
Sva hefir brodir Gvnlavgr ok sagt j latínv
\subsection{Explicit}
ok eptir þat hvarf þessi syn fra bresti\begin{biblfree} Ólafur Halldórsson, \xref{EA-A3}{Óláfs saga Tryggvasonar en mesta} III  65  (Ed. D\textsuperscript{1})\end{biblfree} \end{msItem} \begin{msItem} \textbf{53vb:32-39}:  \textit{Frá Gunnlaugi ok Oddi}
\subsection{Incipit}
Sva segía brędr gvnlavgr ok oddr
\subsection{Explicit}
þar sem gízorí þotti þess vid {\hskip1pt}\\{} þvrfa ∴\begin{bibitemlist}{1}
 \bibitem {bibitem-17}\xref{FMS3}{Fornmanna sögur} III  173:1-13  (Ed. S)
 \bibitem {bibitem-18}Ólafur Halldórsson, \xref{EA-A3}{Óláfs saga Tryggvasonar en mesta} III  66  (Ed. D\textsuperscript{1})

\end{bibitemlist}
\end{msItem}
\section{Physical description}

\subsection{Object}

\subsubsection{Support description}

\paragraph{Support}
\par
Parchment.
\paragraph{Extent}
\par
53. Fols 32v and 33r are blank. 327mm x 236mm.\par
\par
Foliated in the upper right-hand corner of the recto-pages.
\paragraph{Collation}
\par
The manuscript consists of five extant gatherings: \begin{itemize}
\item[1] I: Fols \textbf{1-10}:  make the first extant quire consisting of five, apparently intact, bifolia.
\item[2] II: Fols \textbf{11-22}:  make a complete quire of six bifolia.
\item[3] III: Fols \textbf{23-37}:  comprise seven bifolia and a singleton, fol. 28, the sixth leaf of the gathering. Col. 32rb has only 13 lines, the rest of the page and fols 32v and 33r are left blank by the scribe. It can be presumed that between fols 32 and 33 there was another blank leaf which made a pair with fol. 28 and was at some stage severed from it.
\item[4] IV: Fols \textbf{38-47}:  make the present quire of five bifolia. There is a lacuna between fols 37 and 38, presumably only a single leaf.
\item[5] V: Fols \textbf{48-53}:  make three pairs of conjoint leaves.
\end{itemize} \par
The manuscript is not well-preserved; the beginning is defective and there are lacunae after fols 10, 37 and 47. The extant leaves have all suffered in some degree from wear and damp. Many passages are consequently hard to decipher, but there has been no loss of text as such except on fol. 37, where the vellum has crumbled away leaving a large hole in the outer column. The vellum used for the codex was not particularly well prepared. Many leaves have holes or gashes in them, some of them large, which were sewn together before the scribe began the work.
\subsubsection{Layout}
\par
The manuscript is written in double columns each approx. 250mm x 84mm with 39 to 42 lines. Majuscules occur in varying colours; shades of yellow, red, green and blue. Traces of gold are detectable here and there, e.g. on fols 3r, 22rb, 29rb, though probably not from true gold-leaf. \par
Most chapter-titles or division marks are in red, but a few are in blue or blue-black with a greenish tint. In some places, instead of chapter-titles, we find examples of a fish sketched in, and occasionally a kind of spiralling line, e.g. on fol. 3. A single example of a small foliage motive occurs on fol. 2vb:21.
\subsubsection{Hands}

\subsubsection{Decoration}
\par
\begin{itemize}
\item[1] Fol. \textbf{15vb}:  has a major initial, a \textit{littera florissa} of a {\bfseries Þ} of four lines with an extension down the left margin.
\item[2] Fol. \textbf{29rb}: : The opening of \textit{Norna-Gests þáttr} contains a \textit{littera florissa} of the letter {\bfseries Þ} with dots and branches. The first three lines are set in, but the round bow of the {\bfseries Þ} extends up the upper margin, while the descendent runs down to the bottom of the margin. The bar is metamorphosing down the page from a dotted line to some sort of a branch with leaf-like outgrows and is ending in two spirals, and a leaf in the middle, in the lower margin. The top of the {\bfseries Þ} is ending in one single spiral.
\item[3] Fol. \textbf{31va}: The opening of \textit{Helga þáttr Þórissonar} contains a major initial, a \textit{littera florissa} of a {\bfseries Þ}. 3 lines are set in, but the letter extends in the margin. The {\bfseries Þ} ends with two spirals in the lower margin.
\item[4] Fol. \textbf{33va}: The opening of \textit{Færeyinga saga} contains a major initial, a \textit{littera florissa} of the letter {\bfseries M} with branches and leaves. The four top lines are set in, but the {\bfseries M} extends in the upper margin. The right-hand line of the {\bfseries M} just intrudes on of the first letters of the text, showing that the illumination was done after the scribe finished his work.
\item[5] Fol. \textbf{46rb}: contains a major initial a {\bfseries Þ} with branches and leaves, with an extention down the margin with one fleur-de-lis-like outgrow, and ends with a fleur-de-lis.
\item[6] Fol. \textbf{51va}: : The opening of \textit{Halldórs þáttr Snorrasonar} contains a major initial, a \textit{littera florissa} of the letter {\bfseries H}. Four lines are set in, but the initial extends nine lines up the margin.
\end{itemize}  \par
There are also some little drawings in the margins, e.g. fol. \textbf{10r}: : A man with a fish on the hook of his fishing rod; fol. \textbf{22r}: : A kissing couple; fol. \textbf{26r}: : Three knights fighting a dragon who is eating one of them.\par
There are a few proper marginalia in the codex and some of them are illegible, some can only be read in part. \begin{itemize}
\item Fol. \textbf{26r}:  ‘Gifur(?) þu kv(?) {\hskip1pt}\\{} sem bezt sem {\hskip1pt}\\{} heiter frut{\hskip1pt}\\{}titum’.  {\small\itshape [Note: the reading of the last word is uncertain.]} 
\item Fol. \textbf{27r}:  between the columns: ‘íon {\hskip1pt}\\{} hef{\hskip1pt}\\{}vr g{\hskip1pt}\\{}ert og(?) {\hskip1pt}\\{} til {\hskip1pt}\\{} sett {\hskip1pt}\\{} [...] {\gap }br{\hskip1pt}\\{}ef’ (sixteenth century).
\item Fol. \textbf{35v}:  below the text: ‘Ek einar’ (fifteenth century).
\item Fol. \textbf{36r}:  below the text: ‘Austr ædd’ (c. 1500).
\item Fol. \textbf{38r}:  below the text: ‘ave maria’ (c. 1500).
\item Fol. \textbf{46r}:  below the text: ‘olafur [...] {\gap } {\hskip1pt}\\{} ok [...] {\gap }’ (fifteenth century).
\item The originally blank column 32rb has some nib-trials, including the alphabet, and there is more writing, pen trials, epistolary forms, apophthegms and the like, on the pages originally left blank, fols 32v and 33r.
\item At the top of fol. 32v a hand from c. 1400 or the first half of the fifteenth century wrote: ‘þeim godvm monnvm sem þetta bref sía edr heyra sennder sira gvdvardr prestr {\hskip1pt}\\{} þorkeli þormrmodzsyní kvedív gvds ok sína kvnnígt gerandi’.
\item At the top of fol. 32v below the note just mentioned another hand has written: ‘Þeim er eckí þat til mozt er þannen lifa sem wílía hallda síg med heídr ok gozs ok harma fra ser skílía∴ {\hskip1pt}\\{} Hʀafn aurn hane mr þerna haukur stelkur spoe ualur ok kraka. langleítt haufut quad gyllta sá kautt <j> myr⟨kri⟩’. The stanza is from \textit{Skáld-Helga rímur} but the list of bird-names and the adage are not known from elsewhere.
\item Parts of these entries are repeated in younger, unpractised hands in other spaces on fols 32v and 33r, as if it had been set them as a copybook model.
\item Fol. \textbf{32v}:  ‘þoruardur skrifar illa’ (c. 1500).
\item A little further down the page: ‘helge h.’
\item Fol. \textbf{33r}:  ‘halur er erlaus þiofs haus’ (sixteenth century), now very faint and perhaps partly scraped off. Elsewhere on the page ‘halur’ is written on its own.
\item Fol. \textbf{33v}:  in a rather clumsily script with matching spelling, most probably from the sixteenth century: ‘vertv at bok þine {\hskip1pt}\\{} bal fvsa son vel þa skaltu {\hskip1pt}\\{} fara til stehiaríns.’
\item Fol. \textbf{33v}:  Another entry, doubtless made at or about the same time, and slightly more literate than the first, reads: ‘þeim godvm monnum vertu at bok þine ion narfa son medan ek {\hskip1pt}\\{} er aburtv ellegerar fær þv fleinging’.
\end{itemize} 
\subsection{Binding}
\par
In his catalogue,  {\ref AM 394 fol.},  {\name Jón Sigurðsson} says that AM 62 fol. was bound in a pasteboard binding. This was probably the work of  {\name Mattias Larsen Bloch} done at some time in the years 1771-73.\par
The manuscript was then re-bound three times, first, in the 1880'es and later, in 1934 by  {\name Carl Lund}. During conservation from 9 March 1981 to 10 July 1984, the manuscript was rebound by  {\name Birgitte Dall} in a modern standard half-binding.
\subsection{Accompanying material}
There is an AM-slip pasted in front of the volume reading: ‘Þetta fragment true eg hafa {\hskip1pt}\\{} heyrt Skalholltz kirkiu til. {\hskip1pt}\\{} iafnvel þott þad eigi stande {\hskip1pt}\\{} i neinu afhendingar registre. {\hskip1pt}\\{} Eitt quer her ur feck eg ur ỏd{\hskip1pt}\\{}rum stad enn resten. {\hskip1pt}\\{} Sira Olafur 1699. kallade þad...’
\section{History}

\subsection{Origin}
\par
The manuscript was written in Iceland.  {\name Kristian Kålund} (\textit{KatalogI41}) dated the manuscript to the fifteenth century. Later, however,  {\name Stefán Karlsson} (\textit{Ritun Reykjarfjarðarbókar130}) dated it to the end of the fourteenth century. Ólafur Halldórsson (\textit{The Saga of King Olaf Tryggvason18}) means that the manuscript probably was not written later than c. 1370-80.
\subsection{Provenance}
\par
According to Ólafur Halldórsson (\textit{The Saga of King Olaf Tryggvason18}), a note written by bishop  {\name Oddur Einarsson} in 1612, now in  {\ref AM 416 a 4to}, fol. 6v, should concern AM 62 fol. If this is true, Oddur borrowed the manuscript from  {\name Magnús Hjaltason}, and then he in turn lent it to  {\name Grimur Ormsson}. When Oddur borrowed the manuscript from Magnús, it was in poor shape, but was ruined when it was returned to Oddur: ‘Magnus Hialltaßon hefur {\hskip1pt}\\{} fyrer lỏngu ljed mier Olafs {\hskip1pt}\\{} sỏgur lasnar Þær liede {\hskip1pt}\\{} eg Grijme Ormßyne hier {\hskip1pt}\\{} heima vmm veturenn til jdku{\hskip1pt}\\{}nar þa hann var hia mjer {\hskip1pt}\\{} og fordiarfade hann þær suo {\hskip1pt}\\{} ad eg hef ecke getad þeim aptur {\hskip1pt}\\{} skilad þuij þær voru lasnar {\hskip1pt}\\{} dur. og fundust kueren {\hskip1pt}\\{} eptter honum aptur og framm {\hskip1pt}\\{} enn þad sem epter er af þeim {\hskip1pt}\\{} er hier til synes hef eg {\hskip1pt}\\{} tid þetta fyrer Magnuse {\hskip1pt}\\{} og hefur hann lofad þad skyllde {\hskip1pt}\\{} kuitt þo hann feinge þær all{\hskip1pt}\\{}drej apttur þuj þær være {\hskip1pt}\\{} lijtels verdar. þo eru þær {\hskip1pt}\\{} obitaladar af mier til’. \par
Two names are written on fol. 33v: They are those of  {\name Páll Fúsason (Vigfússon)} (‘bal fvsa son’) and  {\name Jón Narfason} (‘ion narfa son’), who were doubtless boys when they wrote the sentences. They can be identified as Páll Vigfússon, later \textit{lögmaður} and living at  {\name Hlíðarendi} in  {\name Fljótshlíð} (1511-70) who was son of  {\name Vigfús Erlendsson} \textit{lögmaður} (d. 1521). Vigfús had a brother, Narfi, and Narfi’s son Jón must be the other youngster named on fol. 33v of the manuscript. Since the writer of the stanza from \textit{Skáld-Helga rímur} and what follows it on fol. 32v was also the scribe of the document in  {\ref AM Fasc. XIII 1}, it is evident that the codex was in  {\name Eyjafjörður} region at the time and probably at  {\name Mýrka} in  {\name Hörgárdalur} when the document was written in November 1451. Further support for the location of the manuscript in the region around Eyjafjörður may be found in the occurrence of the name Guðvardur on fol. 32v.  {\name Eyjafjarðarsýsla} appears to be the only part of  {\name Iceland} where this name was in use.
\subsection{Acquisition}
\par
 {\name Árni Magnússon} gives the following information about the acquisition of the manuscript on a slip at the front of the volume: ‘Þetta fragment true eg hafa {\hskip1pt}\\{} heyrt Skalholltz kirkiu til. {\hskip1pt}\\{} iafnvel þott þad eigi stande {\hskip1pt}\\{} i neinu afhendingar registre. {\hskip1pt}\\{} Eitt quer her ur feck eg ur ỏd{\hskip1pt}\\{}rum stad enn resten. {\hskip1pt}\\{} Sira Olafur 1699. kallade þad...’ The \textit{sira} Ólafur reference is to a list of manuscripts which  {\name Jón Vídalín}, bishop of  {\name Skálholt}, sent to Árni in 1699. \par
The list, written by \textit{sira}  {\name Ólafur Jónsson} (1672-1702), no longer exists but an extract from it in Árni Magnússon's hand is in  {\ref AM 435 a 4to}, fol. 154r-v: ‘Anno 1699. sende Mag. Jon Th.s. {\hskip1pt}\\{} mier Registur yfir nockrar kalf{\hskip1pt}\\{}skinns skrædur, giỏrt af Sira Olafi Jons {\hskip1pt}\\{} syne, sem hann qvadst mier ut{\hskip1pt}\\{}vegad geta. Þar i bland voru {\hskip1pt}\\{} fragmenta af Olafs Sỏgu Tryggva{\hskip1pt}\\{}sonar in folio. (a)  a] Þetta fragment eignadist eg sidan, er i {\hskip1pt}\\{} storu folio. Item feck eg ur ỏdrum {\hskip1pt}\\{} stad ä Islande nockud sem heyrde þar {\hskip1pt}\\{} til, og lagde eg þad hier sammanvid’. \par
On fol. 57v in AM 435 a 4to, also in Árni's hand, is the following: ‘Fragmentum af Olafs Sỏgu Tryggvasonar, i storu folio: hefur til forna, öefad, heyrt Skalholltzkirkiu til. Eg feck þetta fragment i tveim stỏdum, ä Islande, nockud þar af Mag. Jone Widalin, og nockud ur ỏdrum stad.’ \par
An inventory of property of Skálholt was made when  {\name Þórður Þorláksson} succeeded as bishop in 1674. From it Árni Magnússon produced a list of the Icelandic books the cathedral of Skálholt owned at that time, now found on fols 153 and 156 of AM 435a 4to. No. 6 in the list is \textit{Olafs saga Helga}, here Árni denies that the codex belonged to Skálholt: ‘Þetta mun vera einhvernveiginn mis{\hskip1pt}\\{}skrifad, þvi eingin Olafs Helga Saga {\hskip1pt}\\{} hefur fylgt Sklholltz kirkiu so mikid  sem eg hefe skynia orded. Kynne {\hskip1pt}\\{} vera villt mlum, og eiga ad vera {\hskip1pt}\\{} Olafs Tryggvasonar Sỏgu frag{\hskip1pt}\\{}mentum, kanskie þad sem er i {\hskip1pt}\\{} stőru folio, og kynne þ hafa fyll{\hskip1pt}\\{}ra vered, þő þad sie og ővïst. {\hskip1pt}\\{} Endelega kynne þeir sem afhend{\hskip1pt}\\{}ïnguna giỏrdu, hafa lited skagt til, {\hskip1pt}\\{} og tekid qvi pro qvo.’ \par
Ólafur Haldórsson (\textit{The Saga of King Olaf Tryggvason18}) suggests that it is possible that the codex originally contained the sagas of both kings, \textit{Ólafs saga helga} and \textit{Ólafs saga Tryggvasonar}, and that in 1674 it still had at least parts of \textit{Ólafs saga helga} in it. He asserts that this surmise finds support in the note written by Bishop  {\name Oddur Einarsson} in 1612 (see above).
\subsection{Additional}
\par
Katalogiseret 19 Oktober 1999 af  {\name EW-J}.\par
During the restoration 9 March 1981 to 10 July 1984 by  {\name Birgitte Dall} the leaves were restored and set on meeting guards and the manuscript was rebound in a modern standard half binding.\par
The manuscript was photographed twice, the first time in 1968 and for the second time 6 August 1992. The second set of photographs was presumably made for the facsimily-edition.\par
Supplementary photographs of fols 29-32r and 37, 38r.\par
In August 1992 photographs in uv-light were taken of fols 1r, 5v, 6r, 9r, 10v, 11r-v, 13r, 20v, 21r-v, 23r, 24v, 25r, 26r, 30r, 33v, 34r, 37v, 38r, 43r, 51v, and 53v.\par
\begin{bibitemlist}{1}
 \bibitem {bibitem-24}\textit{plate}  plade 133 6 August 1992.
 \bibitem {bibitem-25}\textit{plate} plade 21 s.d.  (Supplementary photographs of fols 29-32r, 37, 38r.)
 \bibitem {bibitem-26}\textit{b/w prints}  AM 62 fol. 6. August 1992
 \bibitem {bibitem-27}\textit{b/w prints} AM 62 fol. s.d.  (supplementary photographs of fols 29-32r and 37, 38r.)
 \bibitem {bibitem-28}\textit{b/w prints}  AM 62 fol August 1992  (Supplementary photogrpahs in uv-light of fols 1r, 5v, 6r, 9r, 10v, 11r-v, 13r, 20v, 21r-v, 23r, 24v, 25r, 26r, 30r, 33v, 34r, 37v, 38r, 43r, 51v, and 53v.)
 \bibitem {bibitem-29}\textit{diapositive} AM 29 fol. s.d.  (fol. 29r)

\end{bibitemlist}
\begin{bibitemlist}{1}
 \bibitem {bibitem-30}Kålund, \xref{KKKat}{Katalog} I  41-42
 \bibitem {bibitem-31}Ólafur Halldórsson, \xref{EIMF20}{The saga of King Olaf Tryggvason: AM 62 fol.} 9-26  (Facsimile-edition)
 \bibitem {bibitem-32}Ólafur Halldórsson, \xref{OIHalld1963}{Úr sögu skinnbóka}  (With facsimile of fol. 32ra:14-35)
 \bibitem {bibitem-33}Stefán Karlsson, \xref{StefKarl14}{Ritun Reykjarfjarðarbókar}

\end{bibitemlist}

\end{document}
